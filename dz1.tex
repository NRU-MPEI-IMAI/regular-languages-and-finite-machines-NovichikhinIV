\documentclass[a4paper,12pt]{article}

%%% Работа с русским языком
\usepackage{cmap}					% поиск в PDF
\usepackage{mathtext} 				% русские буквы в формулах
\usepackage[T2A]{fontenc}			% кодировка
\usepackage[utf8]{inputenc}			% кодировка исходного текста
\usepackage[english,russian]{babel}	% локализация и переносы

%%% Дополнительная работа с математикой
\usepackage{amsfonts,amssymb,amsthm,mathtools} % AMS
\usepackage{amsmath}
\usepackage{icomma} % "Умная" запятая: $0,2$ --- число, $0, 2$ --- перечисление


\usepackage[notransparent]{svg}     % svg

% Использование симовла лямюда в рисунках svg 
\usepackage[utf8]{inputenc}
\DeclareUnicodeCharacter{03BB}{$\lambda$}


%% Шрифты
\usepackage{euscript} % Шрифт Евклид
\usepackage{mathrsfs} % Красивый матшрифт

%% Перенос знаков в формулах (по Львовскому)
\newcommand*{\hm}[1]{#1\nobreak\discretionary{}
{\hbox{$\mathsurround=0pt #1$}}{}}

%%% Работа с картинками
\usepackage{graphicx}  				% Для вставки рисунков
\graphicspath{{svg/}}      % папки с картинками
\setlength\fboxsep{3pt} 			% Отступ рамки \fbox{} от рисунка
\setlength\fboxrule{1pt}			% Толщина линий рамки \fbox{}
\usepackage{wrapfig} 				% Обтекание рисунков и таблиц текстом

%%% Работа с таблицами
\usepackage{array,tabularx,tabulary,booktabs} % Дополнительная работа с таблицами
\usepackage{longtable} 						  % Длинные таблицы
\usepackage{multirow} 						  % Слияние строк в таблице


%%% Заголовок
\author{Новичихин И.В.}
\title{ДЗ №1: Регулярные языки и конечные автоматы}
\date{\today}

\begin{document}

\maketitle
\newpage


\section{Задание №1. Построить конечный автомат, распознающий язык}

$ 1) \; L_1=\{\omega\in\{a,b,c\}^* : |\omega|_c = 1 \} $ \\

\includesvg{graph_1_1.dot.svg} \\




$ 2) \; L_2=\{\omega\in\{a,b\}^* : |\omega|_a \leqslant 2, |\omega|_b \geqslant 2 \} $ \\

Разделим язык $L_2$ на два языка 
$A=\{\omega\in\{a,b\}^* : |\omega|_a \leqslant 2 \} $ и $B=\{\omega\in\{a,b\}^* : |\omega|_b \geqslant 2 \} $

\includesvg{graph_1_2_1.dot.svg}\\
\includesvg{graph_1_2_2.dot.svg}

$L_2=A \times B$\\ 
Множество конечных состояний в $L_2$: 13, 23, 33\\
Таблица переходов:
\begin{center}
    \begin{tabular}{ |c|c|c|c| } 
        \hline
        $A$ & $B$ & переход по $a$ & переход по $b$ \\
        \hline
        1 & 1 & 21 & 12 \\
        \hline
        1 & 2 & 22 & 13 \\
        \hline
        1 & 3 & 23 & 13 \\
        \hline
        2 & 1 & 31 & 22 \\
        \hline
        2 & 2 & 32 & 23 \\
        \hline
        2 & 3 & 33 & 23 \\
        \hline
        3 & 1 & - & 32 \\
        \hline
        3 & 2 & - & 33 \\
        \hline
        3 & 3 & - & 33 \\
        \hline
    \end{tabular}
\end{center}

 Автомат для $L_1$:
 
\includesvg[scale=0.7]{graph_1_2_3.dot.svg}




3) $L_3=\{\omega\in\{a,b\}^*:|\omega|_a \neq |\omega|_b \} $ \\

Этот язык не получится описать ДКА, потому что автоматы беспамятные <> , то есть не получится запомнить в ДКА разное ли количество символов a и b \\


4)  $L_4=\{\omega\in\{a,b\}^* : \omega \omega = \omega \omega \omega \} $ \\
        
\includesvg[scale=0.7]{graph_1_4.dot.svg}






\section{ Задание №2. Построить конечный автомат, используя прямое произведение}

1) $ L_1=\{\omega\in\{a,b\}^* : |\omega|_a \geqslant 2 \wedge |\omega|_b \geqslant 2 \} $ \\

$A=\{\omega\in\{a,b\}^* : |\omega|_a \geqslant 2 \} $ \\
$B=\{\omega\in\{a,b\}^* : |\omega|_b \geqslant 2 \} $ \\

\includesvg{graph_2_1_1.dot.svg}\\
\includesvg{graph_2_1_2.dot.svg} 

$L_1=A \times B$, \\
$\Sigma=\{a,b\}$, \\
$s=\{11\}$, \\
$T=\{33\}$. \\
переходы:\\
\begin{center}
    \begin{tabular}{ |c|c|c|c| } 
        \hline
        $A$ & $B$ & переход по $a$ & переход по $b$ \\
        \hline
        1 & 1 & 21 & 12 \\
        \hline
        1 & 2 & 22 & 13 \\
        \hline
        1 & 3 & 23 & 13 \\
        \hline
        2 & 1 & 31 & 22 \\
        \hline
        2 & 2 & 32 & 23 \\
        \hline
        2 & 3 & 33 & 23 \\
        \hline
        3 & 1 & 31 & 32 \\
        \hline
        3 & 2 & 32 & 33 \\
        \hline
        3 & 3 & 33 & 33 \\
        \hline
    \end{tabular}
\end{center}

\includesvg[scale=0.7]{graph_2_1_3.dot.svg} 



2) $ L_2=\{\omega \in\{a,b\}^* : |\omega| \geqslant 3 \wedge |\omega| \text{ нечётное} \} $ \\
$A=\{\omega \in\{a,b\}^* : |\omega| \geqslant 3\} $ \\
$B=\{\omega \in\{a,b\}^* : |\omega| \text{ нечётное} \} $


\includesvg[scale=0.7]{graph_2_2_1.dot.svg}\\
\includesvg{graph_2_2_2.dot.svg}

$L_2=A \times B$\\
$\Sigma=\{a,b\}$, \\
$s=11$,\\
$T=\{33\}$.\\
переходы:\\
\begin{center}
    \begin{tabular}{ |c|c|c| } 
        \hline
        $A$ & $B$ & переход по $a$ или $b$ \\
        \hline
        1 & 1 & 22 \\
        \hline
        1 & 2 & 21 \\
        \hline
        2 & 1 & 32 \\
        \hline
        2 & 2 & 31 \\
        \hline
        3 & 1 & 42 \\
        \hline
        3 & 2 & 41 \\
        \hline
        4 & 1 & 42 \\
        \hline
        4 & 2 & 41 \\
        \hline
    \end{tabular}
\end{center}

\includesvg[scale=0.7]{graph_2_2_3.dot.svg} \\

ДКА можно упростить до следующего вида: \\

\includesvg[scale=0.7]{graph_2_2_4.dot.svg} \\



3) $L_3=\{\omega \in\{a,b\}^* : |\omega|_a \text{ чётно} \wedge |\omega|_b \text{ кратно } 3 \} $ \\
$A=\{\omega \in\{a,b\}^* : |\omega|_a \text{ чётно} \} $ \\
$B=\{\omega \in\{a,b\}^* : |\omega|_b \text{ кратно } 3 \} $:

        
\includesvg{graph_2_3_1.dot.svg}\\
\includesvg{graph_2_3_2.dot.svg}

$L_3=A \times B$, \\
$\Sigma=\{a,b\}$, \\
$s=11$,\\
$T=\{11\}$. \\
переходы:\\
\begin{center}
    \begin{tabular}{ |c|c|c|c| } 
        \hline
        $A$ & $B$ & переход по $a$ & переход по $b$ \\
        \hline
        1 & 1 & 21 & 12 \\
        \hline
        1 & 2 & 22 & 13 \\
        \hline
        1 & 3 & 23 & 11 \\
        \hline
        2 & 1 & 11 & 22 \\
        \hline
        2 & 2 & 12 & 23 \\
        \hline
        2 & 3 & 13 & 21 \\
        \hline
    \end{tabular}
\end{center}

\includesvg[scale=0.7]{graph_2_3_3.dot.svg} 



4)  $L_4= \neg L_3 $ \\

$T_4= Q_3 \setminus T_3 = \{ 12, 13, 21, 22, 23 \} $

всё остальное такое же как в $L_3$\\

$\Sigma=\{a,b\}$, \\
$s=11$,\\
переходы:\\
\begin{center}
    \begin{tabular}{ |c|c|c|c| } 
        \hline
        $A$ & $B$ & переход по $a$ & переход по $b$ \\
        \hline
        1 & 1 & 21 & 12 \\
        \hline
        1 & 2 & 22 & 13 \\
        \hline
        1 & 3 & 23 & 11 \\
        \hline
        2 & 1 & 11 & 22 \\
        \hline
        2 & 2 & 12 & 23 \\
        \hline
        2 & 3 & 13 & 21 \\
        \hline
    \end{tabular}
\end{center}


\includesvg[scale=0.7]{graph_2_4_1.dot.svg} 







5) $ L_5= L_2 \setminus L_3 $ \\

$ L_5= L_2 \setminus L_3 = L_2 \cap \neg L_3 = \neg L_3 \times L_2 $, \\

$\Sigma=\{a,b\}$, \\
$s=\langle11,11\rangle$, \\
$T=\{\langle12,42\rangle, \langle13,42\rangle, \langle21,42\rangle, \langle22,42\rangle,\langle23,42\rangle\}$\\

переходы для $L_5$: 
\begin{center}
    \begin{tabular}{ |c|c|c|c|c| } 
        \hline
        $\neg L_3$ & $L_2$ & переход по $a$ & переход по $b$ \\
        \hline
        11 & 11 & 21, 22 & 12, 22 \\
        \hline 
        11 & 22 & 21, 31 & 12, 31 \\
        \hline
        11 & 31 & 21, 42 & 12, 42 \\
        \hline
        11 & 42 & 21, 41 & 12, 41 \\
        \hline
        11 & 41 & 21, 42 & 12, 42 \\
        \hline
        12 & 11 & 22, 22 & 13, 22 \\
        \hline
        12 & 22 & 22, 31 & 13, 31 \\
        \hline
        12 & 31 & 22, 42 & 13, 42 \\
        \hline
        12 & 42 & 22, 41 & 13, 41 \\
        \hline
        12 & 41 & 22, 42 & 13, 42 \\
        \hline
        13 & 11 & 23, 22 & 11, 22  \\
        \hline
        13 & 22 & 23, 31 & 11, 31 \\
        \hline
        13 & 31 & 23, 42 & 11, 42 \\
        \hline
        13 & 42 & 23, 41 & 11, 41 \\
        \hline
        13 & 41 & 23, 42 & 11, 42 \\
        \hline
        21 & 11 & 11, 22 & 22, 22 \\
        \hline
        21 & 22 & 11, 31 & 22, 31 \\
        \hline
        21 & 31 & 11, 42 & 22, 42 \\
        \hline
        21 & 42 & 11, 41 & 22, 41 \\
        \hline
        21 & 41 & 11, 42  & 22, 42 \\
        \hline
        22 & 11 & 12, 22 & 23, 22 \\
        \hline
        22 & 22 & 12, 31 & 23, 31 \\
        \hline
        22 & 31 & 12, 42 & 23, 42 \\
        \hline
        22 & 42 & 12, 41 & 23, 41 \\
        \hline
        22 & 41 & 12, 42 & 23, 42 \\
        \hline
        23 & 11 & 13, 22 & 21, 22 \\
        \hline
        23 & 22 & 13, 31 & 21, 31 \\
        \hline
        23 & 31 & 13, 42 & 21, 42 \\
        \hline
        23 & 42 & 13, 41 & 21, 41 \\
        \hline
        23 & 41 & 13, 42 & 21, 42 \\
        \hline
    \end{tabular}
\end{center} 


\rotatebox{90}{\includesvg[scale=0.49]{graph_2_5_1.dot.svg} } \\





\section{ Задание №3. Построить минимальный ДКА по регулярному выражению}

1) $ (ab + aba)^{*}a $

Сначала составим недетерминированный конечный автомат, а потом детерминированный

\includesvg[scale=0.6]{graph_3_1_1.dot.svg} \\

\begin{center}
    \begin{tabular}{ |c|c|c| } 
        \hline
         & a & b  \\
        \hline
        1 & 3, 6, 10 & \\
        \hline
        3, 6, 10 &  & 4, 7 \\
        \hline
        4, 7 & 3, 6, 8, 10 & \\
        \hline
        3, 6, 8, 10 & 3, 6, 10 & 4, 7 \\
        \hline
    \end{tabular}
\end{center} 

ДКА

\includesvg[scale=0.6]{graph_3_1_2.dot.svg} \\


2) $ a(a (ab)^{*} b)^{*} (ab)^{*} $

НКА

\includesvg[scale=0.35]{graph_3_2_1.dot.svg} \\

Построим эквивалентный ДКА\\
\begin{center}
    \begin{tabular}{ |c|c|c| } 
        \hline
        $Q$ & $a$ & $b$ \\
        \hline\hline
        0 & 1,2,9,10,13 & - \\
        \hline
        1,2,9,10,13 & 3,4,7,11 & - \\
        \hline
        3,4,7,11 & 5 & 2,8,9,10,12,13 \\
        \hline
        5 & - & 4,6,7 \\
        \hline
        2,8,9,10,12,13 & 3,4,7,11 & - \\
        \hline
        4,6,7 & 5 & 2,8,9,10,13 \\
        \hline
        2,8,9,10,13 & 3,4,7,11 & -\\
        \hline
    \end{tabular}
\end{center}

\includesvg[scale=0.45]{graph_3_2_2.dot.svg} \\

\{1,2,9,10,13\}, \{2,8,9,10,12,13\}, \{2,8,9,10,13\} эквивалентны \\

\{3,4,7,11\}, \{4,6,7\} эквивалентны \\

\includesvg[scale=0.6]{graph_3_2_3.dot.svg} \\

3) $ (a + (a + b)(a + b)b)^{*} $

НКА

\includesvg[scale=0.9]{graph_3_3_1.dot.svg} \\

Построим эквивалентный ДКА\\

\begin{center}
    \begin{tabular}{ |c|c|c| } 
        \hline
        $Q$ & $a$ & $b$ \\
        \hline\hline
        1 & 12 & 2 \\
        \hline
        12 & 123 & 23 \\
        \hline
        2 & 3 & 3 \\
        \hline
        123 & 123 & 123 \\
        \hline
        23 & 3 & 13 \\
        \hline
        3 & - & 1 \\
        \hline
        13 & 12 & 12 \\
        \hline
    \end{tabular}
\end{center}

\includesvg[scale=0.9]{graph_3_3_2.dot.svg} \\

4) $ (b + c) ((ab)^{*}c + (ba)^{*})^{*} $

НКА

\includesvg[scale=0.35]{graph_3_4_1.dot.svg} \\

Построим эквивалентный ДКА\\

{\footnotesize
\begin{center}
    \begin{tabular}{ |c|c|c|c| } 
        \hline
        $Q$ & $a$ & $b$ & $c$ \\
        \hline\hline
        1,2,4 & - & 3,6,7,8,9,12,14,15,18,19,20 & 5,6,7,8,9,12,14,15,18,19,20 \\
        \hline
        3,6,7,8,9,12,14,15,18,19,20 & 10 & 16 & 7,8,9,12,13,14,15,18,19,20 \\
        \hline
        5,6,7,8,9,12,14,15,18,19,20 & 10 & 16 & 7,8,9,12,13,14,15,18,19,20 \\
        \hline
        10 & - & 9,11,12 & - \\
        \hline
        16 & 7,8,9,12,14,15,17,18,19,20 & - & - \\
        \hline
        7,8,9,12,13,14,15,18,19,20 & 10 & 16 & 7,8,9,12,13,14,15,18,19,20 \\
        \hline
        9,11,12 & 10 & - & 7,8,9,12,13,14,15,18,19,20 \\
        \hline
        7,8,9,12,14,15,17,18,19,20 & 10 & 16 & 7,8,9,12,13,14,15,18,19,20 \\
        \hline

    \end{tabular}
\end{center}
}

\rotatebox{90}{\includesvg[scale=0.5]{graph_3_4_2.dot.svg} } \\

уберем эквивалентные вершины:

\includesvg[scale=0.55]{graph_3_4_3.dot.svg} \\

5) $ (a + b)^{+}(aa+bb+abab+baba)(a+b)^{+} $

НКА

\includesvg[scale=0.45]{graph_3_5_1.dot.svg} \\

минимальный ДКА

\includesvg[scale=0.7]{graph_3_5_2.dot.svg} \\









\section{ Задание №4. Определить является ли язык регулярным или нет}

1) $ L=\{(aab)^{n}b(aba)^{m} : n \geqslant 0, m \geqslant 0\} $

Язык является регулярным, построим ДКА \\

\includesvg[scale=0.9]{graph_4_1_1.dot.svg} \\

2) $ L = \{uaav : u \in \{a, b\}^*, \; v \in \{a, b\}^*, |u|_b \geqslant |v|_a\} $ \\

Применим лемму о разрастании. 
Зафиксируем $\forall n \in \mathbb{N} $ 
и рассмотрим слово $\omega = b^{n}aaa^{n}, \; |\omega| = 2n + 2 \geq n$. Теперь рассмотрим все разбиения этого слова $\omega = xyz$ такие, что $|y| \neq 0, \; |xy| \leq n$:

$x = b^{k}, $
$y = b^{l}, $ 
$z = b^{n - k - l}aaa^n,$

где $1 \leq k + l \leq n \; \wedge \; l > 0$

Других разбиенний, удовлетворяющих данным условиям, нет.
Для любого из таких разбиений слово $xy^0z \notin L$. 
Лемма не выполняется, значит, $L$ не регулярный язык. \\



3) $L = \{a^mw : w \in \{a, b\}^{*}, \; 1 \geqslant |w|_b \geqslant m\}$ \\

Применим лемму о разрастании. 
Зафиксируем $\forall n \in \mathbb{N} $ 
и рассмотрим слово $\omega = a^nb^n, \; |\omega| = 2n \geqslant n$. 
Теперь рассмотрим все разбиения этого слова $\omega = xyz$ такие, что $|y| \neq 0, \; |xy| \leq n$:

$x = a^{l}, \; y = a^{m}, \; z = a^{n-l-m}b^{n},$ 

где $l + k \leqslant n \; \wedge \; m \ne 0$ \\

Других разбиенний, удовлетворяющих данным условиям, нет. 
Накачка: \\

$ xy^{0}z = a^{l}(a^{m})^{0}a^{n-l-m}b^{n} = a^{n-m}b^{n} \notin L, \; i 
\geqslant 0 \in \mathbb{N} $

Лемма не выполняется, значит, $L$ не регулярный язык. \\




4)  $ L = \{a^{k}b^{m}a^{n} : k = n \vee m > 0\} $

Применим лемму о разрастании. 
Зафиксируем $\forall n \in \mathbb{N} $ 
и рассмотрим слово $\omega = a^nba^n, \; |\omega| = 2n + 1 \geqslant n$. 
Теперь рассмотрим все разбиения этого слова $\omega = xyz$ такие, что $|y| \neq 0, \; |xy| \leq n$:

$ x = a^{i}, \; y = a^{j}, \; z = a^{n-i-j}ba^{n}, $

где $i + j \leqslant n $ \\

Других разбиенний, удовлетворяющих данным условиям, нет.
Накачка:
$ xy^{k}z = a^{i}(a^{j})^{k}a^{n-i-j}ba^{n} = a^{n+j(k-1)}ba^{n} \notin L, \; k 
\geqslant 2 \in \mathbb{N} $
Получили противоречие, значит лемма не выполняется, следовательно, $L$ не регулярный язык. \\



5) $ L = \{ucv : u \in \{a, b\}^*, \; v \in \{a, b\}^*, u \ne v^R \} $

Рассмотрим отрицание языка $L$ \\
$\overline L = \{ ucv \ | \ u \in \{ a,b \}^*, v \in \{ a,b\}^* , u = v^R \}$ \\

Применим лемму о разрастании.
Зафиксируем $\forall n > 0$. 
Возьмем слово $w = a^{n} c a^{n}, |w| = n + 1 + n \geq n$.

Пусть $ 0 < i < n$. Тогда составим разбиение:

$ x = a^{n - i} $ \\
$ y = a^i $ \\
$ z = c a^n $ \\

$\forall k \geq 0$,  $xy^kz \in L$.

$ xy^kz = a^{n - i} a^{ik} c a^n = a^{n - i + ik} c a^n = a^{n + i(k-1)} c a^n $

При $k > 1$ условие $ u = v^R $ не выполняется. Значит язык $\overline L$ не регулярный 

Следовательно и язык $L$ не регулярный.







\end{document}
